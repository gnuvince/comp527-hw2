\documentclass[11pt]{article}

\usepackage{float}
\usepackage{fullpage}
\usepackage{listings}
\usepackage{amsmath}
\usepackage{palatino}
\usepackage[utf8]{inputenc}
\usepackage[T1]{fontenc}
\usepackage[english]{babel}
\usepackage{amssymb}
\usepackage{parskip}
\usepackage{url}
\usepackage{graphicx}
\usepackage{proof}
\usepackage{xcolor}


\newcommand{\nand}{\bar{\wedge}}

\begin{document}

\title{COMP527 - Logic and Computation \\ Homework 2}
\author{
  Vincent Foley-Bourgon <vincent.foley-bourgon@mail.mcgill.ca> \\
  Carl Patenaude-Poulin <carl.patenaudepoulin@mail.mcgill.ca>
}

\maketitle

\section{Exercise 1}

Propositions 1, 4, and 6 have proofs in Tutch, see {\it question1.tut} for
the solutions.  Propositions 2, 3, and 5 cannot be proved using
first-order logic.  We will show their partial derivation trees, and
show where the proof cannot continue.

\subsection{Proposition 2}

\[
\infer[\supset I^u]
  {[\exists x:\tau. A \supset P(x)] \supset [A \supset \forall x:\tau. P(x)]}
  {
    \infer[\supset I^v]
      {A \supset \forall x:\tau. P(x)}
      {
         \infer[\exists E^{w,a}]
            {\forall x:\tau. P(x)}
            {
              \infer[u]{\exists x:\tau. A \supset P(x)}{} &
              \infer[\forall I^b]
                 {\forall x:\tau. P(x)}
                 {
                    \deduce{P(b)}{?}
                 }
            }
      }
  }
\]



In the upper-right branch, we cannot use the variable $b$ to properly
introduce the {\it for all} connective.



\subsection{Proposition 3}

\[
\infer[\supset I^u]
  {[(\forall x:\tau. P(x)) \supset A] \supset [\exists x:\tau. P(x) \supset A]}
  {
    \infer[\exists I]
      {\exists x:\tau. P(x) \supset A}
      {
        \infer{a:\tau}{} &
        \infer[\supset I^v]
          {P(a) \supset A}
          {
            \infer[\supset E]
              {A}
              {
                \infer[u]{(\forall x:\tau. P(x)) \supset A}{} &
                \infer[\forall I^b]
                  {\forall x:\tau. P(x)}
                  {
                    \deduce{P(b)}{\mathcal{?}}
                  }
              }
          }
      }
  }
\]

Since we cannot prove $P(b)$, this proof cannot be completed.

\subsection{Proposition 5}

\[
\infer[\supset I^u]
  {\neg(\forall x:\tau. B(x)) \supset \exists x:\tau. \neg B(x)}
  {
    \infer[\exists I]
      {\exists x:\tau. \neg B(x)}
      {
        \infer{b:\tau}{} &
        \infer[\supset I^v]
          {\neg B(b)}
          {
            \infer[\supset E]
              {\bot}
              {
                \infer[u]{\neg(\forall x:\tau. B(x))}{} &
                \infer[\forall I^c]
                  {\forall x:\tau. B(x)}
                  {
                    \deduce{B(c)}{?}
                  }
              }
          }
      }
  }
\]

Same thing as in the previous proposition, we cannot prove $B(c)$.

\section{Exercise 2}

See the file {\it question2.tut} for the proofs and proof terms.


\section{Exercise 3}

\subsection{Task 1}
% -- introducing the isomorphism f between A and B in scope y
% let f = (fn (a:A) => (... : B) and fn (b:B) => (... : A)) in y
% -- using the left elimination rule; binding the name `a` to (f b) in y
% let a = f b in y
% -- using the right elimination rule; binding the name `b` to (f^(-1) a) in y
% let (f b) = a in y
\[
\infer[\equiv I^{u,v}]
   { \langle a \to \phi_1; b \to \phi_2 \rangle: \mathbf{A \equiv B}}
   {
     \infer*{\phi_1:\mathbf{B}}{\infer[u]{a:\mathbf{A}}{}} &
     \infer*{\phi_2:\mathbf{A}}{\infer[v]{b:\mathbf{B}}{}}
   }
\]

\[
\infer[\equiv E_l]
  {\phi_1:\mathbf{B}}
  {
    \langle a \to \phi_1; b \to \phi_2 \rangle: \mathbf{A \equiv B} &
    a:\mathbf{A}
  }
\]

\[
\infer[\equiv E_r]
  {\phi_2:\mathbf{A}}
  {
     \langle a \to \phi_1; b \to \phi_2 \rangle: \mathbf{A \equiv B} &
    b:\mathbf{B}
  }
\]


\subsection{Task 2}

\subsubsection{Reduction rules}

\begin{align*}
   \langle a \to \phi_1; b \to \phi_2 \rangle x &\Longrightarrow_r [x/a]\phi_1: B \\
   \langle a \to \phi_1; b \to \phi_2 \rangle^{-1} y) &\Longrightarrow_r [y/b]\phi_2: A
\end{align*}

\subsubsection{Expansion rules}

\begin{align*}
 &\langle a_1 \to \phi_1; b_1 \to \phi_2  \rangle
 \\&\Longrightarrow_e\\
 &\langle a_1 \to  \langle a_2 \to \phi_1; b_2 \to \phi_2 \rangle~a_1 ;
         b_1 \to  \langle a_2 \to \phi_1; b_2 \to \phi_2 \rangle ^{-1}~b_1
 \rangle
\end{align*}

\subsection{Task 3}

\subsubsection{Proof on $\equiv E_l$}

\[
\mathcal{D} =
\infer
  {[\Gamma, x:A, \Gamma'] \vdash C}
  {
    \deduce{[\Gamma, x:A, \Gamma'] \vdash B \equiv C}{\mathcal{D}_1} &
    \deduce{[\Gamma, x:A, \Gamma'] \vdash B}{\mathcal{D}_2}
  }
\]

\begin{align*}
  & \Gamma \vdash A \text{ : by assumption} \\
  & [\Gamma, \Gamma'] \vdash B \equiv C \text{ : by i.h. on $\mathcal{D}_1$ and assumption} \\
  & [\Gamma, \Gamma'] \vdash B \text{ : by i.h. on $\mathcal{D}_2$ and assumption} \\
  & [\Gamma, \Gamma'] \vdash C \text{ : by $\equiv E_l$}
\end{align*}


\subsubsection{Proof on $\equiv E_r$}

\[
\mathcal{D} =
\infer
  {[\Gamma, x:A, \Gamma'] \vdash B}
  {
    \deduce{[\Gamma, x:A, \Gamma'] \vdash B \equiv C}{\mathcal{D}_1} &
    \deduce{[\Gamma, x:A, \Gamma'] \vdash C}{\mathcal{D}_2}
  }
\]

\begin{align*}
  & \Gamma \vdash A \text{ : by assumption} \\
  & [\Gamma, \Gamma'] \vdash B \equiv C \text{ : by i.h. on $\mathcal{D}_1$ and assumption} \\
  & [\Gamma, \Gamma'] \vdash C \text{ : by i.h. on $\mathcal{D}_2$ and assumption} \\
  & [\Gamma, \Gamma'] \vdash B \text{ : by $\equiv E_r$}
\end{align*}

\subsubsection{Proof on $\equiv I^{v,w}$}

\[
\mathcal{D} =
\infer
  {[\Gamma, u:A, \Gamma'] \vdash B \equiv C}
  {
    \deduce{[\Gamma, u:A, \Gamma', v:B] \vdash C}{\mathcal{D}_1} &
    \deduce{[\Gamma, u:A, \Gamma', w:C] \vdash B}{\mathcal{D}_2}
  }
\]


\begin{align*}
  & \Gamma \vdash A \text{ : by assumption} \\
  & [\Gamma, \Gamma', v:B] \vdash C \text{ : by i.h using $\mathcal{D}_1$ and assumptions} \\
  & [\Gamma, \Gamma', w:C] \vdash B \text{ : by i.h using $\mathcal{D}_2$ and assumptions} \\
  & [\Gamma, \Gamma'] \vdash B \equiv C \text{ : by $\equiv I^{v,w}$}
\end{align*}



\section{Exercise 4}

\subsection{$\wedge I$}
\[
\infer
  {\Gamma \vdash A \wedge B = C \wedge D}
  {
    \deduce{\Gamma \vdash \langle M,N \rangle : A \wedge B}{E_1} &
    \deduce{\Gamma \vdash \langle M,N \rangle : C \wedge D}{E_2}
  }
\]


\begin{align*}
  & \Gamma \vdash M:A \text{ by } \wedge I \\
  & \Gamma \vdash M:C \text{ by } \wedge I \\
  & \Gamma \vdash A=C \text{ by i.h. on $E_1$ and $E_2$} \\
  & \Gamma \vdash M:B \text{ by } \wedge I \\
  & \Gamma \vdash M:D \text{ by } \wedge I \\
  & \Gamma \vdash B=D \text{ by i.h. on $E_1$ and $E_2$} \\
  & \Gamma \vdash A \wedge B = C \wedge D
\end{align*}

\subsection{$\wedge E_l$}
\[
\infer
  {\Gamma \vdash A = C}
  {
    \deduce{\Gamma \vdash M: A \wedge B}{E_1} &
    \deduce{\Gamma \vdash M: C \wedge D}{E_2}
  }
\]


\begin{align*}
  & \Gamma \vdash \text{fst } M : A \text{ by } \wedge E_l \\
  & \Gamma \vdash \text{fst } M : C \text{ by } \wedge E_l \\
  & \Gamma \vdash A = C \text{ by i.h. on $E_1$ and $E_2$}
\end{align*}


\subsection{$\wedge E_r$}
\[
\infer
  {\Gamma \vdash B = D}
  {
    \deduce{\Gamma \vdash M: A \wedge B}{E_1} &
    \deduce{\Gamma \vdash M: C \wedge D}{E_2}
  }
\]


\begin{align*}
  & \Gamma \vdash \text{snd } M : B \text{ by } \wedge E_r \\
  & \Gamma \vdash \text{snd } M : D \text{ by } \wedge E_r \\
  & \Gamma \vdash B = D \text{ by i.h. on $E_1$ and $E_2$}
\end{align*}

\subsection{$\supset I^u$}

\[
\infer
  {\Gamma \vdash A \supset B = C \supset D}
  {
    \deduce{\Gamma \vdash M: A \supset B}{E_1} &
    \deduce{\Gamma \vdash M: C \supset D}{E_2}
  }
\]

\begin{align*}
  & \Gamma \vdash M = \lambda x:A. N : A \supset B \\
  & \Gamma \vdash M = \lambda x:C. N : C \supset D \\
  & [\Gamma, x:A, x:C] \vdash A = C \text { by assumption} \\
  & \Gamma \vdash N: B \text { by } \supset I^x \\
  & \Gamma \vdash N: D \text { by } \supset I^x \\
  & \Gamma \vdash B = D \text { by i.h. on $E_1$ and $E_2$} \\
  & \Gamma \vdash A \supset B = C \supset D
\end{align*}


\subsection{$\supset E$}

\[
\infer
  {\Gamma \vdash B = D}
  {
    \deduce{\Gamma \vdash M N: B}{E_1} &
    \deduce{\Gamma \vdash M N: D}{E_2}
  }
\]


\begin{align*}
  & \Gamma \vdash M : A \supset B \\
  & \Gamma \vdash N : A  \\
  & \Gamma \vdash M : C \supset D \\
  & \Gamma \vdash N : C \\
  & \Gamma \vdash A = C \text{ by i.h. on $E_1$ and $E_2$} \\
  & \Gamma \vdash A \supset B = C \supset D \text{ by proof in previous section} \\
  & \Gamma \vdash M N : B \text{ by } \supset E \\
  & \Gamma \vdash M N : D \text{ by } \supset E \\
  & \Gamma \vdash B = D
\end{align*}


\subsection{$\vee I_l$}
\[
\infer
  {\Gamma \vdash A \vee B = C \vee D}
  {
    \deduce{\Gamma \vdash \text{inl}^B~M: A \vee B}{E_1} &
    \deduce{\Gamma \vdash \text{inl}^D~M: C \vee D}{E_2}
  }
\]


\begin{align*}
  & \Gamma \vdash M : A \text{ by } \vee I_l \\
  & \Gamma \vdash M : C \text{ by } \vee I_l \\
  & \Gamma \vdash A = C \text{ by i.h. on $E_1$ and $E_2$} \\
  & \Gamma \vdash B = D \text{ by assumption} \\
  & \Gamma \vdash A \vee B = C \vee D
\end{align*}

\subsection{$\vee I_r$}
\[
\infer
  {\Gamma \vdash A \vee B = C \vee D}
  {
    \deduce{\Gamma \vdash \text{inr}^A~M: A \vee B}{E_1} &
    \deduce{\Gamma \vdash \text{inr}^C~M: C \vee D}{E_2}
  }
\]


\begin{align*}
  & \Gamma \vdash M : B \text{ by } \vee I_r \\
  & \Gamma \vdash M : D \text{ by } \vee I_r \\
  & \Gamma \vdash B = D \text{ by i.h. on $E_1$ and $E_2$} \\
  & \Gamma \vdash A = C \text{ by assumption} \\
  & \Gamma \vdash A \vee B = C \vee D
\end{align*}


\subsection{$\vee E^{u,v}$}
\[
\infer
  {\Gamma \vdash C = D}
  {
    \deduce{\Gamma \vdash \text{case } M \text{ of inl}^B~x \to N_1 \| \text{inr}^A~y \to N_2 : C}{E_1} &
    \deduce{\Gamma \vdash \text{case } M \text{ of inl}^B~x \to N_1 \| \text{inr}^A~y \to N_2 : D}{E_2}
  }
\]


\begin{align*}
  & \Gamma \vdash N_1: C \text{ by } \vee E^{u,v} \\
  & \Gamma \vdash N_2: C \text{ by } \vee E^{u,v} \\
  & \Gamma \vdash N_1: D \text{ by } \vee E^{u,v} \\
  & \Gamma \vdash N_2: D \text{ by } \vee E^{u,v} \\
  & \Gamma \vdash C = D \text{ by i.h on $E_1$ and $E_2$}
\end{align*}


\subsection{$\top I$}
\[
\infer
  {\Gamma \vdash \top = \top}
  {
    \deduce{\Gamma \vdash M: \top}{E_1} &
    \deduce{\Gamma \vdash M: \top}{E_2}
  }
\]

Trivial.



\end{document}
